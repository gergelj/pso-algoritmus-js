\chapter{Bevezető}

A modern kori mérnöki tudományokban egyre nagyobb teret hódítanak azok a megoldások, amelyek a természet jól bevállt működési elvét, strukturális felépítését imitálják. A biomimetika/bionika egy egész tudományág, ami ezekkel a természetbeli megoldások mérnöki megvalósításával foglalkozik.

A műszaki tudományokban lépten-nyomon szembetaláljuk magunkat optimalizációs problémákkal. Alapvető fizikai törvényszerűségeket is fel lehet állítani, mint optimalizációs probléma: például az energia megmaradási törvénye. Teljesen jogos tehát azt állítani, hogy a természet végzi a legtöbb optimalizációt.

Ha már a természet ilyen jól tud optimalizálni, feltétlenül léteznie kell olyan megoldásoknak, amelyek követnek valamiféle természetes folyamatokat. Ebben a dolgozatban két ilyen megoldás kerül bemutatásra, amely egyszerűen és hatékonyan oldja meg az optimalizációs problémákat: a genetikus algoritmus és a részecske-raj optimalizáció. Ezek az algoritmusok sztochasztikus folyamatok\footnote{A sztochasztikus folyamat, vagy más néven véletlenszerű folyamat, az a folyamat, melyet -- részben vagy teljesen -- valószínűségi változók jellemeznek. Ennek az ellentéte a determinisztikus folyamat, ahol a folyamatot leíró változók nem véletlenszerűek \parencite{wikipedia2020}} \index{sztochasztikus folyamat}, amelyek az evolúciós algoritmusok családjába tartoznak.

Ha a természet olyan összetett folyamatot, mint az optimalizáció képes lebonyolítani, akkor az ember csakis természetes folyamatok által tudja ezt a feladatot leghatékonyabban elvégezni. A fent említett algoritmusok bemutatása és elemzése előtt viszont érdemes megismerkedni az optimalizáció matematikai definíciójával.