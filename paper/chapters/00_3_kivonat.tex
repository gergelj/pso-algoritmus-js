\selectlanguage{hungarian}
\begin{abstract}

A dolgozat célja bemutatni olyan optimalizációs algoritmusokat, melyek létrejöttét a természet inspirálta, valamint kivizsgálni ezen algoritmusok pontosságát. Optimalizációs megoldásoknál nagy szerepet kap a pontosság, így ezek az algoritmusoknak is fő céljuk a hatékony és relatív pontos eredmény biztosítása. A dolgozat a genetikus algoritmust valamint a részecske-raj algoritmust mutatja be, mint az evolúciós algoritmusok két jeles tagját.

Az evolúciós algoritmusok fő előnyei közé tartozik, hogy bármilyen kritériumfüggvényt optimalizálni tudnak, így a matematikai úton lehetetlen eredményt is ki tudják számolni. Közös vonásaik a paramétereikben húzódnak, miszerint a populációszámuk és a ciklusszámuk befolyásolja az algoritmusok pontosságát és gyorsaságát elsősorban. Az evolúciós algoritmusok felhasználási területe többek között a gépi tanulás, mesterséges neuronhálózatok, big data analízis.

A kísérletek lebonyolításához különböző segédeszközöket használtam: a genetikus algoritmushoz a \textit{Matlab} szoftvercsomag \textit{Optimization Toolbox} nevű modulját, míg a részecske-raj optimalizációnál saját kezűleg írt \textit{Python} illetve \textit{Javascript} szoftvert használtam. Kritériumfüggvényként a Rastrigin-féle függvényt választottam. Kísérletenként 100 számítást végeztem, ami végeredményül középértéket, mediánt és szórást számolt ki. A különböző kísérleteket különböző paraméterek mellett ismételtem meg, így összehasonlítható eredményekre jutottam. Az elemzések kimutatták, hogy ezek az algoritmusok – báris a kezdetleges változatuk – nem minden esetben szolgálnak pontos eredménnyel, így kompromisszumos megoldásokat lehet csak ajánlani az algoritmus gyorsasága és pontossága terén.

Kulcsszavak: optimalizáció, evolúciós algoritmusok, genetikus algoritmus, részecske-raj optimalizáció.

\end{abstract}