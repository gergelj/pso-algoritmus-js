\chapter{Evolúciós algoritmusok}

A numerikus módszerek előnye, hogy viszonylag gyorsan, bármilyen függvény esetében tudnak eredménnyel szolgálni, nincsenek különleges követelményeik, mint az analitikus módszerek esetében.

A számítástechnikai irodalomban nagyszámú sztochasztikus optimalizációs algoritmus létezik, egyik csoportjuk ezeknek az evolúciós algoritmusok. Ezeknek az algoritmusoknak az a különlegessége, hogy véletlenszerű változókat tartalmaznak, így két egymás utáni, egyforma feltételek mellett lezajlott számítás nem fog pontosan egyforma eredményt adni.

Az evolúciós algoritmusok, mint ahogy az a nevükben is szerepel, a természetes evolúció működését próbálják imitálni. Közös vonásuk, hogy egyidőben több lehetséges eredménnyel számolnak, ezeknek a halmaza alkotja a \textit{populációt} \index{populáció}. Ez a populáció ciklusról-ciklusra változásokon megy keresztül úgy, hogy minden ciklusban egyre közelebb kerüljön az optimális értékhez.

Ebben a dolgozatban az evolúciós algoritmusok közül két képviselőjük kerül bemutatásra: a genetikus algoritmusok és a részecske-raj algoritmus.
