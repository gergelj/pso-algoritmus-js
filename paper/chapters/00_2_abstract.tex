\selectlanguage{english}
\begin{abstract}
    
The aim of this study is to present optimization algorithms which were inspired by nature and to analyze their accuracy. Accuracy plays an important role in optimization solutions, so the main goal of these algorithms is to ensure efficient and relatively accurate results. The study presents the genetic algorithm and the particle-swarm algorithm as two prominent members of the evolutionary algorithms.

One of the main advantages of evolutionary algorithms is that they can optimize any criterion function even if it’s mathematically impossible to get a result. The evolutionary algorithms’ common features are their parameters: population number and the number of iterations. They influence the accuracy and the speed of the algorithm aswell. Areas of application of these algorithms include machine learning, artificial neural networks, big data analysis.

I used various tools to perform the analysis: for the genetic algorithm I used \textit{Matlab}’s \textit{Optimization Toolbox} package, on the other hand I used my own \textit{Python} and \textit{Javascript} program to evaluate the particle swarm optimization algorithm. I chose the Rastrigin function as the criterion function. Each experiment was performed 100x which ultimately calculated mean, median and standard deviation. I got comparable results after a few sets of these experiments where every set of calculations were carried out with different parameters. These analyzes have shown that these algorithms do not always give accurate results and it is always the best to make compromises in terms of speed and accuracy depending on the optimization problem.

Keywords: optimization, evolutionary algorithms, genetic algorithm, particle swarm optimization.

\end{abstract}