\chapter{Matematikai optimalizáció}

Ebben a fejezetben a matematikai optimalizáció elméleti háttere kerül bemutatásra.

\section{Matematikai optimalizáció definíciója}

Matematikai optimalizáció \index{matematikai optimalizáció} alatt olyan folyamatot értünk, amely egy előre meghatározott kritérium alapján keresi a legjobb megoldást \parencite{vujanovic1980}. Ezt az eredményt hívjuk optimális értéknek. Az optimalizáció kritériumfüggvényét\footnote{Kritériumfüggvény. Szinonímái: költségfüggvény, energiafüggvény, hasznossági függvény.} \index{kritériumfüggvény} a \ref{eq:criteria} képlettel ábrázoljuk.

\begin{equ}[!ht]
  \begin{equation}
    \begin{aligned}
        f: &X \to Y, Y \subseteq R \\
        f &= f(x) \\
        x &= x_1, x_2, ..., x_n
    \end{aligned}
  \end{equation}
  \caption{\label{eq:criteria}}
\end{equ}

Az optimalizáció valós esetben a \ref{eq:criteria} függvény minimális vagy maximális értékének a meghatározását jelenti, vagyis minimum esetében a \ref{eq:min} képlet, maximum esetében a \ref{eq:max} képlet érvényes.

\begin{equ}[!ht]
  \begin{equation}
    f(x^*) \leq f(x), \forall x \in X
  \end{equation}
  \caption{\label{eq:min}}
\end{equ}

\begin{equ}[!ht]
  \begin{equation}
    f(x^*) \geq f(x), \forall x \in X
  \end{equation}
  \caption{\label{eq:max}}
\end{equ}

Mivel a fenti esetekben a függvény legkisebb (illetve legnagyobb) értékét az egész értelmezési tartomány szintjén az $x^*$ adja meg, ezért ezt globális optimalizációnak\footnote{Globális optimalizáció kiterjed az egész értelmezési tartomány szintjére. Lokális optimalizáció esetében csak az $x^*$ közelében vizsgáljuk a függvényt: $\parallel x^* - x \parallel < \varepsilon$} nevezzük.

A kritériumfüggvény mellett más egyenletek (\ref{eq:g} képlet) is szerepelhetnek a probléma meghatározásában.

\begin{equ}[!ht]
  \begin{equation}
    g_k(x) = 0
  \end{equation}
  \caption{\label{eq:g}}
\end{equ}

Ilyenkor korlátos vagy megkötött matematikai optimalizációról beszélünk. A továbbiakban csak megkötés nélküli optimalizációról lesz szó.

\section{Optimalizációs módszerek}

Sokféle analitikus módszer segítségével lehet eljutni a kritériumfüggvény optimális értékéhez. Többségük a függvény többszörös differenciálhatóságát követeli meg, így összetettebb esetekben ezeket a módszereket nem tudjuk alkalmazni. Ahogy nő a probléma dimenzionalitása, úgy válik nehezebbé az analitikus eredmény kiszámítása (ha egyáltalán lehetséges az analitikus eredményig eljutni). Ilyenkor az egyedüli megoldást a numerikus módszerek \index{numerikus módszer} jelentik.

Tegyük fel, hogy a számítógép segítségét kérjük ennek a problémának a megoldásában. Milyen módszerhez kell folyamodnunk, hogy a várt eredményt kapjuk meg? Legrosszabb esetben végigfutnánk az összes elemen, és így megtalálnánk a keresett legjobb értéket. Ez egyrészt nem kivitelezhető, mert a függvényeink általában folytonos térben mozognak, így végtelen számú elemet kellene leellenőrizni. Ha mégis valahogy felosztanánk a függvény dómenjét $N$ részre, az rettenetesen hosszú számítási időt venne igénybe. Egyedüli megoldást a numerikus algoritmusok kínálnak \parencite{rapaic2019}.
